

\colorlet{outlinecolor}{deeppink}

\colorlet{headercolor}{outlinecolor}
\colorlet{rowcolor1}{outlinecolor!70}
\colorlet{rowcolor2}{outlinecolor!50}

\begin{tikzpicture}
	\node [mybox, fill=boxcolor, draw=outlinecolor] (box){%
		\begin{minipage}{0.3\textwidth}
		\vspace{0.1cm}
		
			\underline{Repository name conventions}:
			\begin{itemize}
				\item \textit{Project names} are expected to be \textbf{(a)} a short and descriptive name, \textbf{(b)} in all lowercase letters, and \textbf{(c)} only use dashes as separators.
				\item \textit{Remote repository} is referred to as \inlinebash{origin} in Git.
			\end{itemize}
			\vspace{2mm}
			
			\underline{Commits}: There are some shorthands to refer to the most recently made commits.
			\vspace{-2mm}
			\begin{center}
				\textcolor{background}{
					\begin{tabularx}{\textwidth}{>{\columncolor{rowcolor1}}X|>{\columncolor{rowcolor2}}p{5cm}}
						\arrayrulecolor{boxcolor} % Table line color
						\rowcolor{headercolor} % Header row color
						\multicolumn{1}{c|}{\centering \textbf{Git Command}} & \multicolumn{1}{c}{\centering \textbf{Description}} \\ % Center the header text
						\hline % Add a horizontal line below the header row
						\rowcolor{rowcolor1} \tablebash{HEAD} & Shorthand in Git for the last commit \\
						\rowcolor{rowcolor2} % Color of the second row
						\tablebash{HEAD\^} & Shorthand in Git for the second to last commit  \\
					\end{tabularx}
				}
			\end{center}
			
			\underline{Default branch name}: The Git community is slowly migrating away from the default branch name being \inlinebash{master} and towards it being \inlinebash{main}.
			\vspace{-2mm}
			\begin{center}
				\textcolor{background}{
					\begin{tabularx}{\textwidth}{>{\columncolor{rowcolor1}}X|>{\columncolor{rowcolor2}}p{5cm}}
						\arrayrulecolor{boxcolor} % Table line color
						\rowcolor{headercolor} % Header row color
						\multicolumn{1}{c|}{\centering \textbf{Git Command}} & \multicolumn{1}{c}{\centering \textbf{Description}} \\ % Center the header text
						\hline % Add a horizontal line below the header row
						\rowcolor{rowcolor1} \tablebash{git branch -m main} & Change the default branch name from \tablebash{master} $\rightarrow$ \tablebash{main} for a \textbf{specific repo} \\
						\rowcolor{rowcolor2} % Color of the second row
						\tablebash{git config --global init.defaultBranch main} & Change the default branch across \textbf{all repositories}  \\
					\end{tabularx}
				}
			\end{center}
			
			\vspace{-1mm}
			\underline{Naming new branches}: One convention for branch naming is to have a prefix that states the branch's general purpose. i.e., 
			\vspace{-2mm}
			\begin{center}
				\textcolor{background}{
					\begin{tabularx}{\textwidth}{>{\columncolor{rowcolor1}}X|>{\columncolor{rowcolor2}}p{5.45cm}}
						\arrayrulecolor{boxcolor} % Table line color
						\rowcolor{headercolor} % Header row color
						\multicolumn{1}{c|}{\centering \textbf{Branch Type}} & \multicolumn{1}{c}{\centering \textbf{Convention}} \\ % Center the header text
						\hline % Add a horizontal line below the header row
						\rowcolor{rowcolor1} \tablebash{feature} & Branch name is either \tablebash{feat/my-description} or \tablebash{feature/my-description} \\
						\rowcolor{rowcolor2} % Color of the second row
						\tablebash{bugfix} & Branch name is either \tablebash{fix/my-description} or \tablebash{bugfix/my-description}  \\
						\rowcolor{rowcolor1} 
						\tablebash{release} & Branch name is \tablebash{release/version-number} \\
						\rowcolor{rowcolor2}
						\tablebash{hotfix} & Branch name is \tablebash{hotfix/my-description} \\
					\end{tabularx}
				}
			\end{center}

			
		\end{minipage}
	};
	\node[fancytitle, right=10pt, fill=outlinecolor, text=background, draw=outlinecolor, rounded corners] at (box.north west) {Git Conventions};
\end{tikzpicture}
