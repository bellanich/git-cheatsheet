\colorlet{outlinecolor}{deeppink}

\colorlet{headercolor}{outlinecolor}
\colorlet{rowcolor1}{outlinecolor!70}
\colorlet{rowcolor2}{outlinecolor!50}

\begin{tikzpicture}
	\node [mybox, fill=boxcolor, draw=outlinecolor] (box){%
		\begin{minipage}{0.3\textwidth}
		\vspace{0.1cm}
		
			\underline{Repository name conventions}: what to keep in mind when naming projects and referencing remote
			\begin{itemize}
				\item \textit{Project names} are expected to be \textbf{(a)} a short and descriptive name, \textbf{(b)} in all lowercase letters, and \textbf{(c)} only use dashes as separators.
				\item \textit{Remote repository} is referred to as \inlinebash{origin} in Git.
			\end{itemize}
			\vspace{2mm}
			
			\underline{Commits}: There are some shorthands to refer to the most recently made commits.
			\vspace{-2mm}
			\begin{center}
				\textcolor{background}{
					\begin{tabularx}{\textwidth}{>{\columncolor{rowcolor1}}X|>{\columncolor{rowcolor2}}p{5cm}}
						\arrayrulecolor{boxcolor} % Table line color
						\rowcolor{headercolor} % Header row color
						\multicolumn{1}{c|}{\centering \textbf{Git Command}} & \multicolumn{1}{c}{\centering \textbf{Description}} \\ % Center the header text
						\hline % Add a horizontal line below the header row
						\rowcolor{rowcolor1} \tablebash{HEAD} & Shorthand in Git for the last commit \\
						\rowcolor{rowcolor2} % Color of the row
						\tablebash{HEAD\^} & Shorthand for 2nd to last commit. \\
						\rowcolor{rowcolor1} % Color of the row
						\tablebash{HEAD}$\sim$\tablebash{3} & Shorthand for 3 commits before current \tablebash{HEAD} for \textbf{linear branch histories} \\
						\rowcolor{rowcolor2} % Color of the row
						\tablebash{HEAD\^3} & Traverses a \textbf{non-linear history}, where exact commits back from  \tablebash{HEAD} depends on branching pattern \\
					\end{tabularx}
				}
			\end{center}
			\vspace{-1mm}
			
			\underline{Default branch name}: In 2020, \href{https://github.blog/2020-07-27-highlights-from-git-2-28/#introducing-init-defaultbranch}{the Git community moved away from calling the default branch name from \inlinebash{master} to calling it \inlinebash{main}}.
			\begin{itemize}
				\item \textit{Why?} The term "master" has an \href{https://about.gitlab.com/blog/2021/03/10/new-git-default-branch-name/}{unsavory connotation for historical reasons}.
				\item \textit{Implications.} Some older projects still use \inlinebash{master} as their main branch name, since renaming it is a risky process. 
			\end{itemize}
			
			\vspace{-3mm}

			
		\end{minipage}
	};
	\node[fancytitle, right=10pt, fill=outlinecolor, text=background, draw=outlinecolor, rounded corners] at (box.north west) {Git Conventions};
\end{tikzpicture}

\begin{tikzpicture}
	\node [mybox, fill=boxcolor, draw=outlinecolor] (box){%
		\begin{minipage}{0.3\textwidth}
			\vspace{0.1cm}
			
							
			Reasons for keeping \inlinebash{master} as the default in older project include:
			\begin{itemize}
					\item Renaming it to \inlinebash{main} can cause backwards compatibility issues with earlier project releases
					\item Complicated CI/CD processes can consequently fail in non-obvious ways that are difficult to troubleshoot
					\item Integrations with third-party tools and services can break
			\end{itemize}
			\vspace{0.1cm}
		
			Nonetheless, it is possible to change the default branch name.
			\begin{center}
				\textcolor{background}{
					\begin{tabularx}{\textwidth}{>{\columncolor{rowcolor1}}X|>{\columncolor{rowcolor2}}p{5cm}}
						\arrayrulecolor{boxcolor} % Table line color
						\rowcolor{headercolor} % Header row color
						\multicolumn{1}{c|}{\centering \textbf{Git Command}} & \multicolumn{1}{c}{\centering \textbf{Description}} \\ % Center the header text
						\hline % Add a horizontal line below the header row
						\rowcolor{rowcolor1} \tablebash{git branch -m main} & Change the default branch name from \tablebash{master} $\rightarrow$ \tablebash{main} for a \textbf{specific repo} \\
						\rowcolor{rowcolor2} % Color of the second row
						\tablebash{git config --global init.defaultBranch main} & Change the default branch across \textbf{all repositories}  \\
					\end{tabularx}
				}
			\end{center}
			
			\underline{Naming new branches}: \href{https://dev.to/couchcamote/git-branching-name-convention-cch}{One convention for branch naming} is to have a prefix that states the branch's general purpose. i.e., 
			\vspace{-2mm}
			\begin{center}
					\textcolor{background}{
							\begin{tabularx}{\textwidth}{>{\columncolor{rowcolor1}}X|>{\columncolor{rowcolor2}}p{5.45cm}}
									\arrayrulecolor{boxcolor} % Table line color
									\rowcolor{headercolor} % Header row color
									\multicolumn{1}{c|}{\centering \textbf{Branch Type}} & \multicolumn{1}{c}{\centering \textbf{Convention}} \\ % Center the header text
									\hline % Add a horizontal line below the header row
									\rowcolor{rowcolor1} \tablebash{feature} & Branch name is either \tablebash{feat/my-description} or \tablebash{feature/my-description} \\
									\rowcolor{rowcolor2} % Color of the second row
									\tablebash{bugfix} & Branch name is either \tablebash{fix/my-description} or \tablebash{bugfix/my-description}  \\
									\rowcolor{rowcolor1} 
									\tablebash{release} & Branch name is \tablebash{release/version-number} \\
									\rowcolor{rowcolor2}
									\tablebash{hotfix} & Branch name is \tablebash{hotfix/my-description} \\
								\end{tabularx}
						}
				\end{center}
			\vspace{-3mm}
			
		\end{minipage}
	};
	\node[fancytitle, right=10pt, fill=outlinecolor, text=background, draw=outlinecolor, rounded corners] at (box.north west) {Git Conventions (2)};
\end{tikzpicture}
