%%%%%%%%%%%%%%%%%%%%%%%%%%%%%%%%%%%%%%%%%%%%%%%%%%%%%%%
% MatPlotLib and Random Cheat Sheet
%
% Edited by Michelle Cristina de Sousa Baltazar
%
% http://matplotlib.org/api/pyplot_summary.html
% http://matplotlib.org/users/pyplot_tutorial.html
%
%%%%%%%%%%%%%%%%%%%%%%%%%%%%%%%%%%%%%%%%%%%%%%%%%%%%%%%

\documentclass[a4paper]{article}
\usepackage[landscape]{geometry}
\usepackage{url}
\usepackage{multicol}
\usepackage{amsmath}
\usepackage{amsfonts}
\usepackage{tikz}
\usetikzlibrary{decorations.pathmorphing}
\usepackage{amsmath,amssymb}

\usepackage{colortbl}
\usepackage{mathtools}
\usepackage{amsthm, amsmath, amssymb, amsfonts}
\usepackage{enumitem}

\title{Git Cheat Sheet}
\usepackage[english]{babel}
\usepackage[utf8]{inputenc}
\usepackage{bm}

\newcommand{\pd}[2]{\frac{\partial #1}{\partial #2}}
\newcommand{\loss}[0]{\mathcal{L}}
\newcommand{\chain}[3]{\frac{\partial #1}{\partial #2}\frac{\partial #2}{\partial #3}}
\newcommand{\eq}[1]{\begin{equation*}\begin{split}#1\end{split}\end{equation*}}
\newcommand{\coderef}[0]{Please find the implementation in the folder with the code files.}
\newcommand{\TODO}[1]{\textbf{\textcolor{red}{#1}}}

%%%%%%%%%%%%%%
% Shades of purple colors
%%%%%%%%%%%%%%
\usepackage{xcolor}

% Define colors
\definecolor{background}{HTML}{2D2B55}
\definecolor{text-color}{HTML}{B362FF}
\definecolor{boxcolor}{HTML}{2D2B55}


\definecolor{golden-yellow}{HTML}{FAD000}
\definecolor{highlighter-purple}{HTML}{4D21FC}
\definecolor{bright-green}{HTML}{3AD900}
\definecolor{icy-blue}{HTML}{9EFFFF}
\definecolor{deep-pink}{HTML}{FF628C}


% Apply to document
\pagecolor{background}
\color{text-color}
%%%%%%%%%%%%%%


\advance\topmargin-0.9in
\advance\textheight3in
\advance\textwidth3in
\advance\oddsidemargin-1.5in
\advance\evensidemargin-1.5in
\parindent0pt
\parskip2pt
\newcommand{\hr}{\centerline{\rule{3.5in}{1pt}}}
%\colorbox[HTML]{e4e4e4}{\makebox[\textwidth-2\fboxsep][l]{texto}
\begin{document}

\begin{center}
	\Large{Git Cheat Sheet} \\
\end{center}

\footnotesize
\begin{multicols*}{3}

\tikzstyle{mybox} = [draw=black, fill=white, very thick,
    rectangle, rounded corners, inner sep=10pt, inner ysep=10pt]
\tikzstyle{fancytitle} =[fill=black, text=white, font=\bfseries]

%------------ CONTEÚDO CAIXA RANDOM ---------------



\begin{tikzpicture}
	\node [mybox, fill=boxcolor, draw=icy-blue] (box){%
		\begin{minipage}{0.3\textwidth}
			\underline{Definition}: A family of parametric, non-linear and hierarchical representation learning functions, which are massively optimized with stochastic gradient descent to encode domain knowledge, i.e. domain invariances, stationarity.\\
			\vspace{-3mm}
			\begin{itemize}[leftmargin=4mm]
				\setlength\itemsep{0.0em}
				\item Neural Network is a directed acyclic graph
				% \item Every module can be expressed by $a=h(x;w)$
				\item Use loss function that matches output distribution to improve numerical stability and make gradients larger
				\item Input and output distribution of every module should be the same to prevent inconsistent behavior and harder learning
			\end{itemize}
			\underline{Backprop}: chain rule $\pd{z}{x_i}=\sum_j \chain{z}{y_j}{x_i}$, $\nabla_{\bm{x}} \bm{z} = \left(\pd{\bm{y}}{\bm{x}}\right)^T \cdot \nabla_{\bm{y}} \bm{z}$
			\vspace{-1mm}
			\begin{enumerate}[leftmargin=4mm, label=\arabic*.]
				\setlength\itemsep{0.2em}
				\item Compute forward: $a^{(l)} = h^{(l)}\left(x^{(l)}\right)$, $x^{(l+1)}=a^{(l)}$
				\item Compute reverse: $\pd{\loss}{a^{(l)}} = \left(\pd{a^{(l+1)}}{x^{(l+1)}}\right)^T \cdot \pd{\loss}{a^{(l+1)}}$\\
				$\pd{\loss}{\theta^{(l)}} = \pd{a^{(l)}}{x^{(l+1)}} \cdot \left(\pd{\loss}{a^{(l)}}\right)^T$
				\item Update params: $\theta^{(l)}_{t+1} = \theta^{(l)}_{t}-\eta \nabla_{\theta_t^{(l)}}\loss$
			\end{enumerate}
		\end{minipage}
	};
	\node[fancytitle, right=10pt, fill=icy-blue, text=black] at (box.north west) {Modular Learning};
\end{tikzpicture}


\end{multicols*}
\end{document}