

\colorlet{outlinecolor}{orange}

\begin{tikzpicture}
	\node [mybox, fill=boxcolor, draw=outlinecolor] (box){%
		\begin{minipage}{0.3\textwidth}
			\vspace{0.1cm}
			\textcolor{outlinecolor}{A repository} is a collection of version controlled files that are kept together. This includes \textbf{(a)} all the files related to a specific project/application, \textbf{(b)} the history of changes, and \textbf{(c)} any special configurations. \\
			
			\vspace{-1mm}
			\underline{Git states}: Git has three \textbf{local} states.
			\begin{enumerate}
				\item The \textcolor{outlinecolor}{working directory state} holds all the project or application files. These files may or may not be managed by Git, but Git is aware of them.
				\item The \textcolor{outlinecolor}{staging area state} or \textcolor{outlinecolor}{Git index state} is holding area for the queue of changes to be included in your next commit.
				\item The \textcolor{outlinecolor}{local Git repository state}  is a hidden folder called \inlinebash{.git}, which contains your entire local commit history.
			\end{enumerate}
			Git also has a \textcolor{outlinecolor}{remote (repository) state}, which is just another repository with its own three internal states. \\
			
			\begin{minipage}{\textwidth}
				\centering
				\includegraphics[width=0.5\textwidth]{images/git_stages.png}
				\captionof{figure}{Caption for Image 2.}
			\end{minipage}
			
%			\begin{figure}
%%				\centering
%				\includegraphics[width=0.8\textwidth]{images/git_stages.png}
%				\caption{A short caption for the image.}
%				\label{fig:myimage}
%			\end{figure}
%			
		\end{minipage}
	};
	\node[fancytitle, right=10pt, fill=outlinecolor, text=background, draw=outlinecolor, rounded corners] at (box.north west) {Git Basics};
\end{tikzpicture}
