

\colorlet{outlinecolor}{icypink}

\begin{tikzpicture}
	\node [mybox, fill=boxcolor, draw=outlinecolor] (box){%
		\begin{minipage}{0.3\textwidth}
			\vspace{0.1cm}
			\underline{Source control} is the practice of tracking and managing changes to code. There are two types:
			\begin{enumerate}
				\item In \textcolor{outlinecolor}{centralized source control}, a centralized server acts as the ultimate source of truth for a collection of versioned files. 
				\begin{itemize}
					\item \textit{Implications.} An internet connection to the central server is required for most basic operations.
					\item \textit{Examples.} \href{https://subversion.apache.org}{Subversion}, \href{https://cvs.nongnu.org}{CVS}
				\end{itemize}
				\item \textcolor{outlinecolor}{Distributed or decentralized source control} doesn't require a central source of truth and allows for most operations to be local.
				\begin{itemize}
					\item \textit{Implications.} You can work independently of an internet connection.
					\item \textit{Examples.} Git, \href{https://www.mercurial-scm.org}{Mercurial (Hg)}
				\end{itemize}
			\end{enumerate}
			
			
			
			\underline{History of Git}: \textcolor{outlinecolor}{Git} was \href{https://www.linuxfoundation.org/blog/blog/10-years-of-git-an-interview-with-git-creator-linus-torvalds}{developed by Linus Torvalds}, the creator of Linux, to handle the requirements of \href{https://github.com/torvalds/linux}{the Linux Kernel Project}. It is often used, because...
			\begin{enumerate}
				\item Due to its distributed nature, Git \textcolor{outlinecolor}{can scale massively}.
				\item Git is \textcolor{outlinecolor}{very fast to execute}, since most of its operations are local.
				\item Due to its  history, it has a very \textcolor{outlinecolor}{active community}.
				
			\end{enumerate}
			
%		\vspace{-3mm}
		\end{minipage}
	};
	\node[fancytitle, right=10pt, fill=outlinecolor, text=background, draw=outlinecolor, rounded corners] at (box.north west) {Source Control};
\end{tikzpicture}
